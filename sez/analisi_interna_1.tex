\newpage
\section{Analisi di una pagina interna}
Oltre alla Homepage è stata analizzata anche una pagina interna, nello specifico la pagina di un articolo; tutte le altre pagine di interesse per l'utente (quindi escludendo le pagine Contatti, Collabora e quella pubblicitaria) presentano la stessa struttura.
\subsection{Visione d'insieme}
La pagina di un articolo presenta alcuni elementi coerenti alla homepage. Come nella pagina principale sono presenti infatti l'header con il logo, il menù e i link alle altre pagine, e il footer, il quale però non è di particolare rilevanza neanche in questo caso. \\
%IMMAGINE DI ARTICOLO CON EVIDENZIATO BREADCRUMB
All'apertura della pagina ci si trova davanti alla grande immagine di anteprima dell'articolo: talmente tanto grande che a volte non permette neanche la completa lettura del titolo, e non fornisce quindi alcuna informazione aggiuntiva. Qui di seguito verranno analizzati nel dettaglio tutti gli aspetti.
\subsection{Le 5+1 W}
\paragraph*{Le 5+1 W per le pagine interne} Per quanto riguarda le pagine interne, si può generalmente supporre che l'utente abbia bisogno di meno assi informativi rispetto a quelli necessari per la Homepage. Nello specifico, le domande \textit{When}, \textit{Why} e \textit{How} assumono una rilevanza marginale rispetto a \textit{Where}, \textit{Who} e \textit{What}, i quali restano fondamentali e, anzi, diventano più importanti.

\subsubsection{Where?}
La domanda \textit{dove mi trovo?} trova risposta nel breadcrumb 	presente appena sotto al menù. Esso dovrebbe probabilmente fornire il percorso che ha condotto l'utente all'articolo aperto; essendo però tutti gli articoli passati per la Homepage, il percorso risulta essere sempre\footnote{Non è stato eseguito un test su ogni pagina, di conseguenza è scorretto dire sempre; nonostante questo, è stato provato con più articoli per ogni categoria, e il risultato è sempre quello riscontrato.} lo stesso: \textit{Home}>\textit{Apertura}>\textit{Nome dell'articolo}; di conseguenza questo breadcrumb è inutile.\\
Oltre a questo problema, il testo del Breadcrumb è stato posto in sovrapposizione all'immagine di anteprima: questo lo rende spesso poco visibile, e quindi ancor meno utile. \\

\subsubsection{Who?}
La risposta alla domanda \textit{chi c'è dietro al sito?} coincide con quanto detto riguardo alla homepage. A questo si aggiunge poi il nome dell'autore dell'articolo, il quale è situato appena sotto al nome dell'articolo (ed è perciò necessario fare uno scroll per poterlo leggere).

\subsubsection{What?}
\textit{Cosa mi offre la pagina?} La risposta alla domanda è immediata: nella pagina è possibile trovare l'articolo, e il nome di questo si trova (spesso, ma non sempre) nella prima schermata). 

\subsubsection{Why?}
\textit{Perché sono qui?} Anche in questo caso la risposta è lapalissiana: per leggere l'articolo.

\subsubsection{When?}
\textit{Quali sono le ultime novità?} Essendo la pagina un articolo, le uniche novità possono venire dai commenti. In fondo ad ogni articolo, infatti, è stato implementato un sistema di commenti utilizzando il Social Network Facebook; l'unico contenuto dinamico della pagina, e quindi l'unico cambiamento che l'utente si aspetta, risiede in questo.

\subsubsection{How?}
\textit{Come arrivo a ciò che mi interessa?} Valgono le stesse osservazioni fatte nell'analisi della Homepage.
