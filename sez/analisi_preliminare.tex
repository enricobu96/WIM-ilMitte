\newpage
\section{Analisi Preliminare}
\subsection{Il contesto}
Come anticipato, il sito è rivolto principalmente a italiani emigrati in Germania, turisti e amanti di Berlino; il target corrisponde quindi a queste categorie di utenti. \\ 
Nonostante questo, il sito è accessibile anche a chi è estraneo a questi argomenti, e ciò lo rende quindi un sito abbastanza generalista.

\subsection{Nome del sito e del dominio}
Il nome è un fattore importante per un sito: un nome scelto senza cura rischia di non invogliare l'utente a tornare. Gli errori più grossi che possono essere commessi sono scegliere un nome troppo articolato, difficile da ricordare o troppo simile a quello di altri siti. \\
Nel caso in questione il nome è stato scelto attentamente; sebbene la parola \textit{Mitte} possa rischiare di non dire niente a un utente che non conosce Berlino, lo stesso non si può dire nei confronti del target del sito. \textit{Mitte} è infatti il nome del quartiere centrale della città, ed è il fulcro economico e sociale nonché sede di tutti gli organi di governo della Germania. \\
Oltre al significato evocativo, questo nome ha la peculiarità di essere corto e facile da pronunciare (non possiede simboli non appartenenti all'alfabeto italiano come dieresi ed \textit{eszet}, e la pronuncia italiana è la stessa di quella tedesca), ed è di conseguenza facile da ricordare. \\
Anche il nome del dominio è stato scelto accuratamente, poiché sostanzialmente lo stesso del sito. \\
I criteri di giudizio del nome del sito e del dominio sono:
\begin{itemize}
\item \textbf{Somiglianza con il nome del sito:} il nome del dominio è lo stesso del sito;
\item \textbf{Lunghezza e semplicità del nome:} il nome è composto da due parole di cui un articolo, e la lunghezza totale è di sette caratteri. Non sono presenti lettere straniere né caratteri, quali trattini, che potrebbero inficiare la lettura;
\item \textbf{Unicità del nome:} non esistono altri siti con nomi simili. Attraverso la ricerca di Google, esso si confonde solo con la pagina di Wikipedia e con una sezione del sito \href{http://www.berlino.com}{berlino.com};
\item \textbf{TLD:} il dominio di primo livello è \textbf{.com}, e questo è generalmente uno dei più apprezzati dagli utenti;
\item \textbf{Suono delle parole:} come detto, la pronuncia della parola è semplice, e questo riduce lo sforzo computazionale dell'utente. Il nome, inoltre, inizia con una vocale, e mediamente questa è la scelta più apprezzata dagli utenti.
\end{itemize}
In conclusione, il nome rispetta le linee guida, ed è quindi stata una scelta azzeccata.